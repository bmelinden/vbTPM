\cite{someintroductionghere}
\subsection{The variational approximation}
An exact computation of the Bayesian evidence is impractical or
impossible for most interesting models, and clever approximations are
needed. The approximation we use here is variously known as ensemble
learning, variational Bayes, or (in statistical physics jargong) mean
field theory \cite{mackay2003,bishop2006}, has previously been applied
to biophysical time-series of FRET data
\cite{bronson2009,vandemeent_manuscript,okamoto2012} and \textit{in
  vivo} single particle tracking \cite{persson2013}.  The idea is to
approximate the log evidence by a lower bound, $\ln p(x|N)\geq F_N$,
with
\begin{equation}
  F_N= \int d\theta\sum_{s} q(s)q(\theta)
  \ln\frac{p(x,s|\theta)p(\theta|N)}{q(s)q(\theta)},
\end{equation}
where $q(s)$ and $q(\theta)$ are arbitrary probability distributions
over the hidden states and parameters respectively.  These are
optimized to make the bound as tight as possible for each model, the
model that achieves the highest lower bound wins, and the
corresponding optimal distributions $q(s)q(\theta)$ can be used for
approximate inference about parameter values and hidden states. In
particular, optimizing $F_N$ with respect to the variational
distributions leads to
\begin{align}
  \ln q(\theta)=&-\ln Z_\theta+ \ln p(\theta|N)
  +\mean{\ln p(x,s|\theta)}_{q(s)},\label{seq:qparam}\\
  \ln q(s)=&-\ln Z_s+\mean{\ln p(x,s|\theta)}_{q(\theta)},\label{seq:sparam}
\end{align}
where the $Z$'s are Lagrange multipliers to enforce normalization, and
$\mean{\cdot}_{q(\cdot)}$ denotes an average over $q(\cdot)$. We solve
these equations iteratively until the lower bound converges, repeating
the analysis many times with independent initial conditions in order
to find a global maximum.  The iterative solution approach results in
an EM-type variational algorithm, detailed below.  We refer to
Refs.~\cite{mackay1997,beal2003,persson2013} for details on how to
derive variational algorithms for HMMs, and
Refs.~\cite{beal2003,mackay2003,bishop2006} for more general
discussion of variational inference methods.

\subsubsection{Parameter distributions}
The results of plugging our diffusinve HMM into the parameter update
equation \eqref{seq:qparam} are as follows.  The initial state
probability vector, and each row in the transition matrix (denoted
$A_{j,:}$), are Dirichlet distributed,
\begin{align}
q(\vec{\pi})=&\Dir(\vec{\pi}|\vec{w}^{(\vec{\pi})}),\\
w_j^{(\vec{\pi})}=&\tilde w_j^{(\vec{\pi})}+\mean{\delta_{j,s_1}}_{q(s_{1:T})},
\label{seq:VBMpi}\\
%\end{align}\begin{align}
q(A_{i,:})=&\Dir(A_{i,:}|\vec{w}^{(\matris{A})}),\\
w_{ij}^{(\matris{A})}=&\tilde w_{ij}^{(\matris{A})}
+\sum_{t=2}^T\mean{\delta_{i,s_{t-1}}\delta_{j,s_t}}_{q(s_{1:T})}.\label{seq:VBMAj}
\end{align}
Here, variables under tilde's ($\tilde{~}$) are hyperparameters that
parameterize the prior distributions, and can be interpreted as
pseudo-observations. The Dirichlet density function is
\begin{equation}
  \Dir(\vec \pi | \vec u)=\frac{\Gamma(u_0)}{\prod_j\Gamma(u_j)}
  \prod_j \pi_j^{u_j-1},\quad u_j>1,
\end{equation}
where $u_0=\sum_j u_j$ is called the strength, and the density is
non-zero in the region $0\le\pi_j\le1$, $\sum_j\pi_j=1$ . Before
moving on, we quote some useful expectation values for future
reference,
\begin{align}
  \mean{\ln\pi_i}_{q(\vec{\pi})}=&\psi(w_i^{(\vec{\pi})})-\psi(w_0^{(\vec{\pi})}),
  \label{seq:mlogpi}\\
  \mean{\ln A_{ij}}_{q(\matris{A})}=&\psi(w_{ij}^{(\matris{A})})
  -\psi(w_{i0}^{(\matris{A})}),\label{seq:mlogA}
\end{align}
where $\psi(x)$ is the digamma function, and
\begin{align}
  \mean{\pi_i}_{q(\vec{\pi})}=&\frac{w_i^{(\vec{\pi})}}{w_0^{(\vec{\pi})}},\\
  \Var[\pi_i]_{q(\vec{\pi})}=&
  \frac{w_i^{(\vec{\pi})}\big((1-w_i^{(\vec{\pi})}\big)}{
    (w_0^{(\vec{\pi})})^2\big(1+w_0^{(\vec{\pi})}\big)},\\
  \mean{A_{ij}}_{q(\matris{A})}=&\frac{w_{ij}^{(\matris{A})}}{w_{i0}^{(\matris{A})}},\\
  \Var[A_{ij}]_{q(\matris{A})}=&
  \frac{w_{ij}^{(\matris{A})}\big(1-w_{ij}^{(\matris{A})}\big)}{
    (w_{i0}^{(\matris{A})})^2\big(1+w_{i0}^{(\matris{A})}\big)}.
\end{align}
The bead motion parameters have the following variational distributions
\begin{align}\label{seq:KBtrial}
  q(K_j,B_j)=&\frac{B_j^{n_j}}{W_j}e^{-B_j\big(v_j(K_j-\mu_j)^2+c_j\big)},\\
  W_j=&\frac{c^{-(n_j+\frac 12)}\Gamma(n_j+\frac 12)}{\sqrt{v_j/\pi}},
\end{align}
with the range $B_j\ge 0$, $-\infty<K_j<\infty$. Physically, we might
rather expect $0<K_j<1$, but the extended range for $K_j$ simplifies
the calculations a lot. The VBM equations are
\begin{align}
  n_j=& n_j^0+M_j,\\
  c_j=& c_j^0+C_j+v_j^0(\mu_j^0)^2
  -\frac{\big(v_j^0\mu_j^0+U_j\big)^2}{v_j^0+V_j},\\
  v_j=&v_j^0+V_j,\\
  \mu_j=&\frac{v_j^0\mu_j^0+U_j}{v_j^0+V_j},
\end{align}
with
\begin{align}
  M_j=&\sum_{t=2}^T\mean{\delta_{s_t,j}},\\
  C_j=&\sum_{t=2}^T\mean{\delta_{s_t,j}}\x_t^2,\\
  U_j=&\sum_{t=2}^T\mean{\delta_{s_t,j}}\x_t\cdot\x_{t-1},\\
  V_j=&\sum_{t=2}^T\mean{\delta_{s_t,j}}\x_{t-1}^2.
\end{align}
Some useful expectation values for future reference are
\begin{align}
  \mean{\ln B_j}_{q(\vec{B},\vec{K})}=&\psi\big(n_j+\frac 12\big)-\ln c_j,
  \label{seq:meanBlogB}\\
  \mean{B_j}_{q(\vec{B},\vec{K})}=&\frac{n_j+\frac 12}{c_j},\\
  \mean{B_jK_j^2}_{q(\vec{B},\vec{K})}=&\frac{1}{2v_j}
  +\mu_j^2\frac{n_j+\frac 12}{c_j},\\
  \mean{B_jK_j}_{q(\vec{B},\vec{K})}=&\mu_j\frac{n_j+\frac 12}{c_j},
  \label{seq:meanbK2BK}\\
  \text{Var}[B_j]_{q(\vec{B},\vec{K})}=&\frac{n_j+\frac 12}{c_j^2},\\
  \mean{K_j}_{q(\vec{B},\vec{K})}=&\mu_j,\\ 
  \text{Var}[K_j]_{q(\vec{B},\vec{K})}=&\frac{c_j}{2v_j(n_j-\frac 12)}.
\end{align}
\subsubsection{Hidden state distribution}
The variational distribution has a simple form,
\begin{equation}\label{seq:qgt}
  \ln q(s_{1:T})=-\ln Z+\sum_{t=1}^T\ln h_{s_t}(t)+\sum_{t=2}^T\ln J_{s_{t-1},s_t},
\end{equation}
i.e., an initial state distribution, a point-wise term that depends on
the initial conditions and the data, and a transition
probability. The point-wise 
\begin{equation}
  \ln q(s_{1:T})=-\ln Z+\sum_{t=1}^T\ln h_{s_t}(t)+\sum_{t=2}^T\ln J_{s_{t-1},s_t},
\end{equation}
i.e., an initial state distribution, point-wise terms that depends on
the initial conditions and the data, and transition terms. The
mathematical form of this expression is the same as encountered in
maximum-likelihood optimization of hidden Markov Models, and hence the
normalization constant and expectation values needed for the parameter
update equations can be computed by the Baum-Welch algorithm
\cite{baum1970}, which resembles the transfer matrix solution for spin
models in statistical physics. 

Similarly, and the most likely sequence of hidden states can be
computed by the Viterbi algorithm \cite{viterbi1967}.

Specifically, the initial term is given by
\begin{equation}
  \ln h_j(1)=
  \mean{p(s_1=j|\vec{\pi})}_{q(\vec{\pi})}
  =\psi(w_{j}^{(\vec{\pi})})-\psi(w_0^{(\vec{\pi})}),
\end{equation}
the point-wise contributions for $t>1$ are
\begin{multline}
  \ln h_j(t)=
  \psi\big(n_j+\frac12\big)-\ln(\pi c_j)-\frac{\x_{t-1}^2}{2v_j}\\
-\frac{n_j+\frac12}{c_j}\Bigg(
\x_{t-1}^2\bigg(\mu_j-\frac{\x_t\cdot\x_{t-1}}{\x_{t-1}^2}\bigg)^2\\
+\x_t^2-\frac{\big(\x_t\cdot\x_{t-1}\big)^2}{\x_{t-1}^2}\Bigg),
\end{multline}
and the transition terms are given by
\begin{equation}
    \ln J_{ji}=\psi\big(w_{j,i}^{(\matris{A})}\big)
    -\psi\big(\sum_{k=1}^Nw_{j,k}^{(\matris{A})}\big).
\end{equation}
\subsection{VBEM iterations and model search}
The iterative optimization of the variational distributions are done
as follows. To start with, an initial guess for the variational
parameter distributions are generated. We then alternate between VBE
step, in which we construct the hidden state distribution and compute
the averages $\mean{\delta_{j,s_t}}_{q(s)}$ and
$\mean{\delta_{j,s_t}\delta_{k,s_{t+1}}}_{q(s)}$ in a Baum-Welch
forward-backward sweep, and a VBM step, in which we use these averages
to update the parameter variational distributions, until the lower
bound converges.

The variational approach has the additional useful tendency to
penalizing overfitting already during the VBEM iterations, by
depopulating superfluous states
\cite{mackay1997,beal2003,persson2013}. We exploit this property by
using a greedy search algorithm to explore the model space. The basic
strategy is to start by fitting a model with many states from random
initial conditions, and then exploring less complex models by
gradually removing the least populated states. This saves computing
time by supplying good initial guesses for the low complexity models
(which therefore converge quickly), and by lowering the number of
independent restarts, since it is easier to construct a good initial
guess for a model with many states.

\subsection{The lower bound}
has an especially simple form just after the VBE step
\cite{mackay1997,beal2003,persson2013}, given by the normalization
constant $\ln Z$ of the variational hidden state distribution, minus
the Kullback-Leibler divergences between the variational and prior
parameter distributions,
\begin{multline}
  F=\ln Z
  -\int d\vec{\pi} q(\vec{\pi}) \ln\frac{q(\vec{\pi})}{p(\vec{\pi})}\\
  -\sum_{j=1}^N\Bigg[
  \int d^NA_{j,:}\;q(A_{j,:})\ln\frac{q(A_{j,:})}{p_0(A_{j,:})}\\
  +\int dB_jdK_j\;q(B_j,K_j)\ln\frac{q(B_j,K_j)}{p_0(B_j,K_j)}
  \Bigg].
\end{multline}
The Kullback-Leibler terms can be expressed in terms of the
expectation values computed above. For the initial state distribution,
we get
\begin{multline}
  \int d\vec{\pi} q(\vec{\pi}) \ln\frac{q(\vec{\pi})}{p_0(\vec{\pi})}
  =\ln\tilde w_0^{(\vec{\pi})}
    -\psi(\tilde w_0^{(\vec{\pi})})-\frac{1}{\tilde w_0^{(\vec{\pi})}}\\
    +\sum_{j=1}^N\left[
      \big(w_j^{(\vec{\pi})}-\tilde w_j^{(\vec{\pi})}\big)\psi(w_j^{(\vec{\pi})})
      -\ln\frac{\Gamma(w_j^{(\vec{\pi})})}{\Gamma(\tilde w_j^{(\vec{\pi})})}
      \right].
\end{multline}
To get this simple form, we used that $w_0^{(\vec{\pi})}=1+\tilde
w_0^{(\vec{\pi})}$ (since $\sum_j\mean{\delta_{j,s_1}}=1$), and the
identities $\Gamma(x+1)=x\Gamma(x)$ and $\psi(x+1)=\psi(x)+\frac 1x$.
Furthermore, each row of the transition probability matrix contributes
\begin{multline}
  \int d^NA_{j,:}\;q(A_{j,:})\ln\frac{q(A_{j,:})}{p_0(A_{j,:})}\\
  =\ln\frac{\Gamma(w_{j0}^{(\matris{A})})}{\Gamma(\tilde w_{j0}^{(\matris{A})})}
    -(w_{j0}^{(\matris{A})}-\tilde w_{j0}^{(\matris{A})})\psi(w_{j0}^{(\matris{A})})\\
    -\sum_{k=1}^N\bigg[
      \ln\frac{\Gamma(w_{jk}^{(\matris{A})})}{\Gamma(\tilde w_{jk}^{(\matris{A})})}
      -\big(w_{jk}^{(\matris{A})}-\tilde w_{jk}^{(\matris{A})}\big)
      \psi(w_{jk}^{(\matris{A})})\bigg].
\end{multline}
Finally, the emission parameter of each state contributes
\begin{multline}\label{seq:KBKL}
  \int dB_j\int d^NK_j\;q(B_j,K_j)
  \ln\frac{q(B_j,K_j)}{p(B_j,K_j)}=\ldots\\
  =-\frac{n_j+\frac 12}{c_j}
    \Big(c_j-\tilde c_j-\tilde v_j(\mu_j-\tilde \mu_j)^2\Big)\\
    +\frac 12\ln\frac{v_j}{\tilde v_j}
    +(\tilde n_j+\frac 12)\ln\frac{c_j}{\tilde c_j}
    -\ln\frac{\Gamma\big(n_j+\frac 12\big)}{\Gamma\big(\tilde n_j+\frac 12\big)}\\
    +(n_j-\tilde n_j)\psi\big(n_j+\frac 12\big)
    +\frac{\tilde v_j}{2v_j}
    -\frac 12.
\end{multline}

\subsection{Two types of states}
The above algorithm is readily extended to treat the model where
genuine and spurious states are separated into two different hidden
processes. We implemented a brute force approach to this problem,
where we define new composite hidden states $\hat s_t=(s_t,c_t)$ and
run the above algorithm on this composite model. This has a
significant computational cost, since a simple model with $N_{gen.}$
genuine states and $N_{sp.}$ spurious ones gets
$N_{gen.}\times(1+N_{sp.})$ states after conversion. However, since we
do not perform exhaustive model search in this representation and can
utilize the simpler model to make good initial guesses, this is not a
significant problem.

\subsection{Choice of prior distributions}
We would like to choose uninformative prior distributions in order to
let the data speak for itself as much as possible. This is
unproblematic for the emission parameters $K,B$, since the amount of
data in all states is large enough to overwhelm any prior
influence. We use
\begin{align}
  \tilde\mu_j=&0.6, &
  \tilde n_j=&1,\\
  \tilde v_j =&5.56\text{ nm$^2$},&
  \tilde c_j=&30000\text{ nm$^2$},
\end{align}
which corresponds to 
\begin{align}
  \mean{K_j}=&0.6,&
  \mean{B_j}=&5\times 10^{-5}\text{ nm$^{-2}$},\\
  \mathrm{std}(K_j)=&0.3,&
  \mathrm{std}(B_j)=&141.4\times 10^{-5}\text{ nm$^{-2}$}.
\end{align}
The initial state prior is unproblematic for the opposite reason: the
long length of the trajectories makes the initial state relatively
unimportant to describe the data. We use a constant prior strength of
5,
\begin{equation}
  \tilde w_j^{(\vec\pi)}=5/N,
\end{equation}
where $N$ is the number of hidden states.

The transition probabilities needs more care, because the potentially
low number of transitions per trajectory makes the prior relatively
more influential.  Following \citet{persson2013}, we parameterize this
prior in terms of an expected mean dwell time and an overall number of
pseudocounts (prior strength) for each hidden state.  In particular,
we define a transition \textit{rate} matrix Q with mean dwell time
$t_D$,
\begin{equation}
  Q_{ij}=\frac{1}{t_D}\left(-\delta_{ij}
  +\frac{1-\delta_{ij}}{N-1}\right),
\end{equation}
and then construct the prior based on the transition probability
propagator per unit time step,
\begin{equation}
  \tilde w_{ij}^{(\matris{A})}=\frac{t_Af_{sample}}{n_{downsample}}
  e^{\Delta tQ}.
\end{equation}
Here, $t_A$ is the prior strength; both $t_A$ and $t_D$ is specified
in time units to be invariant a change of sampling frequency. Further,
the timestep is given by $\Delta t=n_{downsample}/f_{sample}$, where
$f_{sample}$ is the sampling frequency (30 Hz in our case), and
$n_{downsample}$ is the downsampling factor (we use 3).  

Numerical experiments by \citet{persson2013} show that choosing the
strength too low compared to the mean dwell time produces a bias
towards sparse transition matrices. This is not desireable in our
case, and we therefore use $t_D=1$ s, and $t_A=5$ s throughout this
work.


{\bf Prior for factorial model: TBA}.\cite{factorialmodelprior}

\subsection{Empirical Bayes update equations}
The empirical Bayes update equations optimizes the lower bound with
respect to the hyperparameters in the prior distribution. This means
optimizing sums of Kullback-Leibler divergence terms. 

The initial state probability, and the rows of the transition
probability matrix, are both Dirichlet distributed. Thus, for $M$
trajectories with Dirichlet parameters $u_j^{(i)}$, $i=1,2,\ldots,M$,
and hyperparameters $\tilde{u}_j$ ($u = w^{(\vec\pi)},u^{(\matris{A})}$),
we need to solve
\begin{multline}
\frac{d}{d \tilde u_j} \sum_i\Bigg( 
 \ln\frac{\Gamma(u_0^{(i)})}{\Gamma(\tilde u_0)}
    -(u_0^{(i)}-\tilde u_0)\psi(u_0^{(i)})\\
    -\sum_{k=1}^N\bigg[
      \ln\frac{\Gamma(u_k^{(i)})}{\Gamma(\tilde u_k^{(i)})}
      -\big(u_k^{(i)}-\tilde u_k\big)
      \psi(u_k^{(i)})\bigg]\Bigg)=0,
\end{multline}
where $u_0^{(i)}=\sum_ku_k^{(i)}$ and similar for $\tilde u_0$. This
leads to the update equations
\begin{equation}
  \psi(\tilde u_0)-\psi(\tilde u_j)
=\frac{1}{M}\sum_i\bigg(
  \psi(u_0^{(i)})-\psi(u_j^{(i)})\bigg).
\end{equation}
A numerical solution turned out to be easier using the variables
$\tilde U_j=\ln \tilde u_j$ (to numerically enforce $\tilde u_j>0$).

For the emission parameters, the update equations are instead derived
from minimizing \Eq{seq:KBKL} summed over $M$ trajectories,
\begin{multline}
  f_{KB}=\sum_i\Bigg(
  -\frac{n^{(i)}+\frac 12}{c^{(i)}} \Big(c^{(i)}-\tilde c-\tilde v(\mu^{(i)}-\tilde
  \mu)^2\Big)\\ +\frac 12\ln\frac{v^{(i)}}{\tilde v} +(\tilde
  n+\frac 12)\ln\frac{c^{(i)}}{\tilde c}
  -\ln\frac{\Gamma\big(n^{(i)}+\frac 12\big)}{\Gamma\big(\tilde n+\frac
    12\big)}\\ +(n^{(i)}-\tilde n)\psi\big(n^{(i)}+\frac 12\big)
  +\frac{1}{2}\Big(\frac{\tilde v}{v^{(i)}} -1\Big)\Bigg).
\end{multline}
Minimizing with respect to $\tilde\mu$ and $\tilde v$ leads to
\begin{align}
   \tilde\mu=&\frac{1}{M}\sum_i\mu^{(i)},\\
   \frac{1}{\tilde v}=&
   \frac{1}{M}\sum_i\Big(\frac{1}{v^{(i)}}+2(\tilde\mu-\mu^{(i)})^2\Big).
\end{align}
The remaining $\tilde c$ and $\tilde n$ lead to
\begin{align}
  \frac{\tilde n+\frac 12}{\tilde c}=&
  \frac{1}{M}\sum_i\frac{n^{(i)}+\frac 12}{c^{(i)}},\\
  \ln\tilde c-\psi\big(\tilde n+\frac 12\big)=&
  \frac{1}{M}\sum_i\bigg(
\ln c^{(i)}-\psi\big(n^{(i)}+\frac 12\big)\bigg),
\end{align}
which we solve numerically. This gets easier by defining
$\alpha=\frac{\tilde n+\frac 12}{\tilde c}$, then solve the second
equation for $\tilde c$ numerically, and finally compute $\tilde
n=\alpha\tilde c-\frac 12$.
