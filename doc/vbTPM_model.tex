\subsection{Diffusive hidden Markov model}
We model the looping dynamics by a discrete Markov process $s_t$ with
$N$ states, a transition probability matrix $\matris{A}$, and initial
state distribution $\vec{\pi}$,
\begin{equation}\label{eq:eoms}
  p(s_t|s_{t-1},\matris{A})=A_{s_{t-1}s_t},\quad p(s_1)=\pi_{s_t}.
\end{equation}
This is the standard hidden part of an HMM, and the physics of TPM
goes into the emission model, that describes the restricted Brownian
motion of the bead. We use a discrete time model of over-damped 2D
diffusion in a harmonic potential, that has been suggested as a
simplified model for TPM \cite{beausang2007b,lindner2013},
\begin{equation}\label{eq:eomx}
  \x_t=K_{s_t}\x_{t-1}+\vec{w}_t/(2B_{s_t})^{1/2},
\end{equation}
where the index $s_t$ indicate parameters that depend on the hidden
state.  Thermal noise enters through the uncorrelated Gaussian random
vectors $\vec{w}_t$ with unit variance. The unintuitive
parametrization is chosen for computational convenience; $K_j$ and
$B_j$ are related to the spring and diffusion constant of the bead,
and some insight into their physical meaning can be gained by noting
that with a single hidden state, Eq.~\ref{eq:eomx} describes a
Gaussian process with zero mean and
\begin{align}
RMS&=\sqrt{\mean{\x^2}}=(B(1-K^2))^{-1/2},\nonumber\\
\frac{\mean{\x_{t+m}\cdot\x_t}}{\mean{\x^2}}&
=K^m\equiv e^{-m\Delta t/\tau},\label{eq:1state}
\end{align}
where $\Delta t$ is the sampling time, and $\tau$ is a bead
correlation time.  This model thus captures the diffusive character of
the bead motion, while still retaining enough simplicity to allow
efficient variational algorithms
\cite{mackay1997,bishop2006}.

\subsection{Factorial model}
We would like to filter out experimental artifacts such as transient
sticking events and tracking errors from the data. To do this, we
introduce a second hidden state $c_t$ that works like an indicator
variable: $c_t=1$ indicates genuine TPM, with bead motion described by
$K_{s_t},B_{s_t}$, but when $c_t>1$, the bead motion is instead
described by parameters $\hat K_{c_t},\hat B_{c_t}$, while $s_t$
evolves completely unseen.  We also allow the transition probabilities
from $c_t=1$ to $c_t>1$ to depend on $s_t$, to allow for the
possibility that sticking events happen more easily in looped than
unlooped states for example. Thus, the hidden states evolve according
to
\begin{equation}
  p(s_{t+1},c_{t+1}|s_t,c_t)=p(s_{t+1}|s_t)p(c_{t+1}|s_t,c_t),
\end{equation}
with $p(s_{t+1}|s_t)=A_{s_ts_{t+1}}$ as usual, and 
\begin{equation}
  p(c_{t+1}|s_t,c_t)=\left\{
  \begin{array}{ll}
    \hat A_{s_tc_{t+1}},&\text{if $c_t=1$,}\\
    \hat R_{c_tc_{t+1}},&\text{if $c_t>1$,}    
  \end{array}\right.
\end{equation}
and initial probabilities 




\cite{vbTPM_model:describefactorialmodel}
