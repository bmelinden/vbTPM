\subsection{Diffusive hidden Markov model}
We model the looping dynamics by a discrete Markov process $s_t$ with
$N$ states, a transition probability matrix $\matris{A}$, and initial
state distribution $\vec{\pi}$,
\begin{equation}\label{eq:eoms}
  p(s_t|s_{t-1},\matris{A})=A_{s_{t-1}s_t},\quad p(s_1)=\pi_{s_t}.
\end{equation}
This is the standard hidden part of an HMM, and the physics of TPM
goes into the emission model, that describes the restricted Brownian
motion of the bead. We use a discrete time model of over-damped 2D
diffusion in a harmonic potential, that has been suggested as a
simplified model for TPM \cite{beausang2007b,lindner2013},
\begin{equation}\label{eq:eomx}
  \x_t=K_{s_t}\x_{t-1}+\vec{w}_t/(2B_{s_t})^{1/2},
\end{equation}
where the index $s_t$ indicate parameters that depend on the hidden
state.  Thermal noise enters through the uncorrelated Gaussian random
vectors $\vec{w}_t$ with unit variance. The unintuitive
parameterization is chosen for computational convenience; $K_j$ and
$B_j$ are related to the spring and diffusion constant of the bead,
and some insight into their physical meaning can be gained by noting
that with a single hidden state, Eq.~\ref{eq:eomx} describes a
Gaussian process with zero mean and
\begin{align}
RMS&=\sqrt{\mean{\x^2}}=(B(1-K^2))^{-1/2},\nonumber\\
\frac{\mean{\x_{t+m}\cdot\x_t}}{\mean{\x^2}}&
=K^m\equiv e^{-m\Delta t/\tau},\label{eq:1state}
\end{align}
where $\Delta t$ is the sampling time, and $\tau$ is a bead
correlation time.  This model thus captures the diffusive character of
the bead motion, while still retaining enough simplicity to allow
efficient variational algorithms
\cite{mackay1997,bishop2006}.

\subsection{Factorial model}
The above algorithm is readily extended to treat the factorial model
where genuine ($s_t$) and spurious ($c_t$) states are separated into
two different hidden processes, with $c_t=1$ indicating a genuine
state (bead motion described by genuine parameters $K_{s_t},B_{s_t}$),
and $c_t>1$ indicating a spurious state (bead motion described by
genuine parameters $\hat K_{c_t},\hat B_{c_t}$). 

The bead motion/emission parameters are exactly analogous to genuine
state parameters in the simple HMM above. For the transitions, we
assume that genuine states evolve independently, and likewise for
spurious states ($c_t>1$), but that the transitions from genuine to
spurious $c_t=1 \to c_{t+1}>1$ depends on the genuine state (see main
text \cite{vbTPMpaper}). This leads to the following model:
\begin{equation}
  p(s_{t+1},c_{t+1}|s_t,c_t)=p(s_{t+1}|s_t)p(c_{t+1}|s_t,c_t),
\end{equation}
with $p(s_{t+1}|s_t)=A_{s_ts_{t+1}}$ as earlier, and
\begin{equation}
  p(c_{t+1}|s_t,c_t)
  =\left\{
  \begin{array}{ll}
    \hat A_{s_tc_{t+1}},&\text{ if $c_t=1$,}\\
    \hat R_{c_tc_{t+1}},&\text{ if $c_t>1$,}\\
  \end{array}
  \right.
\end{equation}
Thus, $\hat{\matris{A}}$ is the transition matrix to induce spurious
states, and $\hat{\matris{R}}$ is the transition matrix to convert
between spurious states and to end them.

vbTPM implements a brute force variational treatment of this extended
model, where we define new composite hidden states $\hat
s_t=(s_t,c_t)$ and modify the simple vb algorithm to run this
composite model, which mainly involves book-keeping to update the
transition count matrices for $\hat{\matris{A}}$, $\hat{\matris{R}}$.
This has a significant computational cost, since a simple model with
$N_{gen.}$ genuine states and $N_{sp.}$ spurious ones gets
$N_{gen.}\times(1+N_{sp.})$ states after conversion. However, since we
do not perform exhaustive model search in this representation and can
utilize the simpler model to make good initial guesses, this is not a
big problem.

