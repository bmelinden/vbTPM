\subsection{Installation}
Get the source code, and make sure VB7, HMMcore, and tools are in your
matlab path, for example by running the script \verb+vbTPMstart+ from
the matlab command prompt. (To add these paths permanently to your
matlab path, see the matlab documentation).

Make sure that HMMcore/ contains binaries for your systems. If not, a
simple matlab compilation script can be found in HMMcore/.

\subsection{Hardware requirements}
Tethered particle motion often produces large data sets of many long
trajectories, which makes the HMM analysis computer intensive. As an
example, one parameter point in our test data sets, about 90
trajectories averaging 45 min, downsampled to 10 Hz, took about 24 h
to go through on two 6 core Intel Xeon E5645 2.40GHz processors. The
analysis time increases sharply with the number of states (including
spurious ones, like transient sticking events).
\subsection{A small test problem}
A small test problem, which runs on a (fast) laptop in about one hour,
with examples of data and runinput files, can be found in example1/,
with the actual data in example1/lacdata/.  The data set has one
calibration (cal) and one production (trj) trajectory for each bead.

\paragraph{Runinput files} 
contain all parameters to run the analysis and access the results. The
meaning of the parameters are documentet in the help text of
\verb+VB7_batch_run.m+, and commented in those files.
\verb+runinput1.m+ refers to an already completed analysis (results in
example1/HMMresults1/), while \verb+runinput2.m+ has not yet run.

\paragraph{Run basic analysis.}
To start analyzing the test data set, type
\verb+VB7_batch_run('runinput2')+ in the Matab command prompt. Since
the runinput file has \verb+one_at_a_time=true;+ this will analyze one
trajectory in the data set. Several calls are needed to complete the
analysis. 

In our experience, matlab tends to hoard memory if several large data
sets are analyzed consecutively.  To work around that, one can use
scripts that starts several matlab sessions with a single trajectory
in each. An example for the bash shell is \verb+runscript1.sh+, which
calls \verb+runinput1.m+. To parallellize, run several instances of
the script at once. \verb+VB7_batch_run+ keeps track of which
trajectories have already been 'checked out', so it is also possible
to run on several computers, it the results folder is synced regularly
(if the same trajectory is checked out multiple times, old results are
simply overwritten).


\paragraph{Manage the analysis.}
\verb+VB7_batch_manage+ is a tool for managing the basic analysis. It
can collect the results and write them to a file, count how many
trajectories in a data set has been analyzed, and also clean up
temporary files from unfinished trajectories, which is useful if an
analysis run is interrupted.

\paragraph{Access the results.} 
The GUI for manual state classification is called
\verb+VB7_batch_postprocess()+. The GUI can be used to inspect the
analysis results in detail, and can also convert the simple HMM models
to factorial models for further analysis. To try it out, use the
runinput file \verb+runinput1.m+, which is already analyzed. To access
the fitted models directly, use \verb+VB7_batch_manage+ with the
'collect' option. The results are returned as cell vectors for
calibration and production trajectories, with the same index structure
as the filenames in the runinput file.


\subsection{Other useful scripts}
