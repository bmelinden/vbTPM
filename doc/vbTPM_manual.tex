\documentclass[11pt,letterpaper,twocolumn]{article}
\usepackage[dvips]{graphicx}
\usepackage{fontspec}
\defaultfontfeatures{Ligatures=TeX}
\usepackage{amsmath}
\usepackage[numbers,sort&compress,round]{natbib}
\usepackage{url}
%\usepackage{epstopdf}
\usepackage{MLHMM}

\usepackage{color}
\usepackage[usenames,dvipsnames]{xcolor}
\newcommand{\jwm}[1]{\textcolor{MidnightBlue}{#1}}
\newcommand{\ml}[1]{\textcolor{BrickRed}{#1}}

%\usepackage{ulem}
\newcommand{\del}[1]{\sout{#1}}

\newcommand{\comment}[1]{{\bf [#1]}}

\newcommand{\parboxc}[1]{\parbox[t]{0.7\columnwidth}{
    \rule[5pt]{0pt}{5pt} \noindent #1 \rule[-3pt]{0pt}{5pt}}}
\newcommand{\parboxcc}[1]{
  \begin{minipage}[t]{0.7\textwidth}
    \rule[5pt]{0pt}{5pt} \noindent{#1} \rule[-3pt]{0pt}{5pt}
  \end{minipage}
%  \parbox[t]{0.7\textwidth}{\rule[5pt]{0pt}{5pt} \noindent #1 \rule[-3pt]{0pt}{5pt}}
}


\title{vbTPM user guide and manual} 

\author{Martin Lind\'en, bmelinden@gmail.com, \today.}
\date{}

\begin{document}
\maketitle
\tableofcontents
% some information that is useful for running the VB7 software for
% analyzing TPM traces.
\section*{To-do list}
Prior distributions. Maybe move some of it to manuscript SI?

Model search.

Factorial model, including priors.

Factorial algorithm.



\section{Getting started with vbTPM}
\subsection{Installation}
Get the source code, and make sure VB7, HMMcore, and tools are in your
Matlab path, for example by running the script \verb+vbTPMstart+ from
the Matlab command prompt. (To add these paths permanently to your
Matlab path, see the Matlab documentation).

Make sure that HMMcore/ contains binaries for your system. If not, a
simple Matlab compilation script can be found in HMMcore/. (See Matlab
documentation for how to set up your mex compiler). For serious
analysis, it is probably faster to compile your own binaries.

\subsection{System requirements}
Tethered particle motion often produces large data sets of many long
trajectories, which makes the HMM analysis computer intensive. As an
example, one parameter point in our test data sets, about 90
trajectories averaging 45 min, downsampled to 10 Hz, took about 24 h
to go through on two 6 core Intel Xeon E5645 2.40GHz processors. The
analysis time increases sharply with the number of states (including
spurious ones, like transient sticking events).

vbTPM is implemented in Matlab, tested on R2012a and R2013a, and
requires the signal processing toolbox.

\subsection{A small test problem}
A small test problem can be found in example1/, with the actual data
in example1/lacdata/.  The data set has one calibration (cal) and one
production (trj) trajectory for each bead, and contains data from five
beads. Using the runinput2.m and runscript2.sh files as described
below, this data set takes less than 30 min to analyze on the above
machine (running all five trajectories in parallell).

\subsubsection*{Runinput files} 
Runinput files contain all parameters to run the analysis and access
the results. The meaning of the parameters are documented in the help
text of \verb+VB7_batch_run.m+, and commented in the runinput files in
example1/.  \verb+runinput1.m+ refers to an already completed analysis
(results in example1/HMMresults1/), while \verb+runinput2.m+ has not
yet run.

\subsubsection*{Run basic analysis}
To start analyzing the test data set, type
\verb+VB7_batch_run('runinput2')+ in the Matab command prompt. Since
the runinput file has \verb+one_at_a_time=true;+ this will analyze one
trajectory in the data set. Several calls are needed to complete the
analysis. 

In our experience, Matlab tends to hoard memory if several large data
sets are analyzed consecutively.  To work around that, one can use
scripts that starts consecutive Matlab sessions and runs a single
trajectory in each.

One example for the bash shell is \verb+runscript1.sh+ and
\verb+runscript1.sh+, which calls \verb+runinput1.m+ (there is also a
\verb+runscript2.sh+).

\subsubsection*{Parallelization}
To parallellize, run several instances of the above script at once,
with \verb+one_at_a_time=true;+ in the runinput
file. \verb+VB7_batch_run+ keeps track of which trajectories have
already been 'checked out', so it is also possible to run on several
computers, if the results folder is synced regularly. If the same
trajectory is checked out multiple times on different computers, old
results are simply overwritten (no harm done if they used the same
runinput file).

\subsubsection*{Manage the analysis}
\verb+VB7_batch_manage+ is a tool for managing the basic analysis. It
can collect the results and write them to a file, count how many
trajectories in a data set has been analyzed, and also clean up
temporary files from unfinished trajectories, which is useful if an
analysis run is interrupted.

\subsubsection*{Access the results} 
The GUI for manual state classification is called
\verb+VB7_batch_postprocess()+. The GUI can be used to inspect the
analysis results in detail, and can also convert the simple HMM models
to factorial models for further analysis. To try it out, use the
runinput file \verb+runinput1.m+, which is already analyzed. To access
the fitted models directly, use \verb+VB7_batch_manage+ with the
'collect' option. The results are returned as cell vectors for
calibration and production trajectories, with the same index structure
as the filenames in the runinput file.

For details on how vbTPM represents the models etc., we refer to
section \ref{sec:notation}.

\subsection{Other useful scripts}
\subsubsection{Data and options}
\paragraph{VB7\_getOptions} 
reads a runinput file and return all variables in a struct.
\paragraph{VB7\_preprocess} 
Converts trajectory data to a format that the analysis code
uses. Input trajectory should be drift-corrected.
\paragraph{BWdriftcorrect}
applies driftcorrection to a position trajectory using a Butterworth-filter.
\paragraph{RMSKBgaussfilter}
computes running averages of RMS and other things, using a Gaussian
kernel filter for smoothing.
\paragraph{VB7\_getTrjData} 
returns the data for a single trajectory in a runinput file in various
formats.

\subsubsection{Models}
\paragraph{VB7\_priorParent} 
A tool to initiate models of various sizes with consistent prior
distributions.
\paragraph{VB7\_initialGuess\_KBregion} 
is a rather complicated function to generate an initial guess for a
model struct (e.g., fill out the M and Mc fields) based on analyzing
the data.
\paragraph{VB7\_GSconversion}
is a tool to create factorial models, by converting genuine states
into spurious ones.
\paragraph{VB7\_removeState}
removes states from a model object.
\paragraph{VB7\_findGenuine}
applies a simple set of rules to determine which states in a given
model are genuine and spurious. An analyzed model for the
corresponding calibration trace i also needed to provide a baseline.
\paragraph{VB7\_inspectStates} 
is a simple tool to navigate in the raw data with the help of a
converged model, for example to take a closer look at hard-to-classify
states.

\subsubsection{VB-EM iterations}
\paragraph{VB7\_VBEMiter} 
is the computational core of vbTPM, and runs a single VB-EM iteration.
\paragraph{VB7iterator} 
runs VB-EM iteratons of a model until convergence.
\paragraph{VB7\_greedySearch}
runs a model search on a single trace from a given initial
guess. Briefly, the search strategy is to systematically remove
low-occupancy states until the lower bound $F$ stops increasing.
\paragraph{VB7\_analyzeTrace}
runs several greedy model searches on single traces. 


\section{Diffusive model for TPM}
\subsection{Diffusive hidden Markov model}
We model the looping dynamics by a discrete Markov process $s_t$ with
$N$ states, a transition probability matrix $\matris{A}$, and initial
state distribution $\vec{\pi}$,
\begin{equation}\label{eq:eoms}
  p(s_t|s_{t-1},\matris{A})=A_{s_{t-1}s_t},\quad p(s_1)=\pi_{s_t}.
\end{equation}
This is the standard hidden part of an HMM, and the physics of TPM
goes into the emission model, that describes the restricted Brownian
motion of the bead. We use a discrete time model of over-damped 2D
diffusion in a harmonic potential, that has been suggested as a
simplified model for TPM \cite{beausang2007b,lindner2013},
\begin{equation}\label{eq:eomx}
  \x_t=K_{s_t}\x_{t-1}+\vec{w}_t/(2B_{s_t})^{1/2},
\end{equation}
where the index $s_t$ indicate parameters that depend on the hidden
state.  Thermal noise enters through the uncorrelated Gaussian random
vectors $\vec{w}_t$ with unit variance. The unintuitive
parametrization is chosen for computational convenience; $K_j$ and
$B_j$ are related to the spring and diffusion constant of the bead,
and some insight into their physical meaning can be gained by noting
that with a single hidden state, Eq.~\ref{eq:eomx} describes a
Gaussian process with zero mean and
\begin{align}
RMS&=\sqrt{\mean{\x^2}}=(B(1-K^2))^{-1/2},\nonumber\\
\frac{\mean{\x_{t+m}\cdot\x_t}}{\mean{\x^2}}&
=K^m\equiv e^{-m\Delta t/\tau},\label{eq:1state}
\end{align}
where $\Delta t$ is the sampling time, and $\tau$ is a bead
correlation time.  This model thus captures the diffusive character of
the bead motion, while still retaining enough simplicity to allow
efficient variational algorithms
\cite{mackay1997,bishop2006}.

\subsection{Factorial model}
To separate genuine looping dynamics from artifacts such as transient
sticking events and tracking errors, we introduce a second hidden
state $c_t$ that works like a dynamic indicator: $c_t=1$ indicates
genuine TPM, with bead motion described by $K_{s_t},B_{s_t}$, while
$c_t>1$ indicates some experimental artifact in action. The bead
motion is then described by parameters $\hat K_{c_t},\hat B_{c_t}$
that depend on $c_t$, while $s_t$ evolves without influencing the
data. 

One might imagine using different types of models for genuine and
spurious states (e.g., transient sticking events might have a
different tethering point than genuine states). We do not explore this
option, but instead classify states based on parameter values, as
described in the main text \cite{maintext}.

We also allow the transition probabilities from $c_t=1$ to $c_t>1$ to
depend on $s_t$, to allow for the possibility that sticking events
happen more easily in looped than unlooped states for example. Thus,
the hidden states evolve according to
\begin{equation}
  p(s_{t+1},c_{t+1}|s_t,c_t)=p(s_{t+1}|s_t)p(c_{t+1}|s_t,c_t),
\end{equation}
with $p(s_{t+1}|s_t)=A_{s_ts_{t+1}}$ as usual, and 
\begin{equation}
  p(c_{t+1}|s_t,c_t)=\left\{
  \begin{array}{ll}
    \hat A_{s_tc_{t+1}},&\text{if $c_t=1$,}\\
    \hat R_{c_tc_{t+1}},&\text{if $c_t>1$,}    
  \end{array}\right.
\end{equation}
and initial probabilities are independent,
\begin{equation}
  p(s_1=j,c_1=k)=\pi_j\hat\pi_k.
\end{equation}

\section{The VB-algorithm}
\subsection{Model selection by maximum evidence}
Our analysis aims not only to extract parameter values from TPM data,
but also to learn the number of hidden states $N$, corresponding to
different DNA-protein conformations. This means that we need to
compare models with different number of unknown parameters.  We take a
Bayesian approach to this problem.

A distinguishing feature of Bayesian data analysis is the treatment of
random variables and unknown parameters on an equal footing
\cite{eddy2004,mackay2003}. 
Hence, given some data $\x_{1:T}$ and a set of competing models with
different number of states $N=1,2,\ldots$ (and \mbox{\textit{1:T}} is
a compact way to denote a whole time series), we can use the laws
of probability to express our confidence about those models in terms
of conditional probabilities,
\begin{equation}
  p(N|\x_{1:T})=p(\x_{1:T}|N)p(N)/p(\x_{1:T}),
\end{equation}
where $p(N)$ expresses our beliefs about the different models prior to
seeing the data, and $p(\x_{1:T})$ is a normalization constant. A
Bayesian rule for model selection is therefore to prefer the model
that maximizes $p(\x_{1:T}|N)$, a quantity known as the evidence. For
our more complex model, parameters and hidden states will have to be
integrated out,
\begin{equation}\label{eq:evidencedef}
  p(\x_{1:T}|N)=\int d\theta\sum_{s_{1:T}} p(\x_{1:T},s_{1:T}|\theta)p(\theta|N),
\end{equation}
where the first factor in the integrand describes the model, and the
second expresses our prior beliefs about the parameters (see
below).

The integrand in the evidence, \Eq{eq:evidencedef}, requires an
explicit expression for the probability of a sequence of bead
positions and hidden states. This expression can be written down based
on the above model, and factorizes in the usual HMM fashion, as
\begin{multline}\label{eq:evidencintegrand}
 p(\x_{1:T},s_{1:T}|\theta)p(\theta|N)%\matris{A},\vec{\pi},\vec{K},\vec{B},N)
=
p(\x_1)p(s_1|\vec{\pi})\\
\times
\prod_{t=2}^{T}p(\x_t|\x_{t-1},s_t,\vec{K},\vec{B})p(s_t|s_{t-1},\matris{A})\\
\times
p(\vec{\pi}|N)
\prod_{j=1}^Np(K_j,B_j|N)p(A_{j,:}|N),
\end{multline}
where $A_{j,:}$ denote row $j$ of the matrix $\matris{A}$.  The first
right hand side line in \Eq{eq:evidencintegrand} describes the initial
state and bead position. We will neglect the factor $p(\x_1)$ from now
on, but the initial state $p(s_1|\vec\pi)$ and transition
probabilities $p(s_t|s_{t-1},\matris{A})$ are given by
\Eq{eq:eoms}, and the bead motion follows from \Eq{eq:eomx},
\begin{equation}
 p(\x_t|\x_{t-1},s_t,\vec{K},\vec{B})
 =\frac{B_{s_t}}{\pi}
 e^{-B_{s_t}(\x_t-K_{s_t}\x_{t-1})^2}.
\end{equation}
Finally, the last line of \Eq{eq:evidencintegrand} contains prior
distributions over parameters conditional on the number of states. We
use conjugate priors, parameterized to have minimal impact on the
inference results (see SI).


%The maximum evidence criterion balances model complexity versus
%goodness of fit in an optimal way, but is expensive to compute
%exactly. Instead, we use an approximation variously known as ensemble
%learning, variational Bayes, or mean field
%theory \cite{mackay2003,bishop2006}, which has previously been applied
%successfully to single molecule FRET data
%\cite{bronson2009,vandemeent_manuscript,okamoto2012} and \textit{in
%  vivo} single particle tracking \cite{persson2013}.  
%Further details are given in the SI.

%\paragraph{Parameters and hidden states}
%{\bf Explain what parameter values and hidden states we plot}. 

\subsection{The variational approximation}
An exact computation of the Bayesian evidence is impractical or
impossible for most interesting models, and clever approximations are
needed. The approximation we use here is variously known as ensemble
learning, variational Bayes, or (in statistical physics jargong) mean
field theory \cite{mackay2003,bishop2006}, has previously been applied
to biophysical time-series of FRET data
\cite{bronson2009,vandemeent_manuscript,okamoto2012} and \textit{in
  vivo} single particle tracking \cite{persson2013}.  The idea is to
approximate the log evidence by a lower bound, $\ln p(x|N)\geq F_N$,
with
\begin{equation}
  F_N= \int d\theta\sum_{s} q(s)q(\theta)
  \ln\frac{p(x,s|\theta)p(\theta|N)}{q(s)q(\theta)},
\end{equation}
where $q(s)$ and $q(\theta)$ are arbitrary probability distributions
over the hidden states and parameters respectively.  These are
optimized to make the bound as tight as possible for each model, the
model that achieves the highest lower bound wins, and the
corresponding optimal distributions $q(s)q(\theta)$ can be used for
approximate inference about parameter values and hidden states. In
particular, optimizing $F_N$ with respect to the variational
distributions leads to
\begin{align}
  \ln q(\theta)=&-\ln Z_\theta+ \ln p(\theta|N)
  +\mean{\ln p(x,s|\theta)}_{q(s)},\label{seq:qparam}\\
  \ln q(s)=&-\ln Z_s+\mean{\ln p(x,s|\theta)}_{q(\theta)},\label{seq:sparam}
\end{align}
where the $Z$'s are Lagrange multipliers to enforce normalization, and
$\mean{\cdot}_{q(\cdot)}$ denotes an average over $q(\cdot)$. We solve
these equations iteratively until the lower bound converges, repeating
the analysis many times with independent initial conditions in order
to find a global maximum.  The iterative solution approach results in
an EM-type variational algorithm, detailed below.  We refer to
Refs.~\cite{mackay1997,beal2003,persson2013} for details on how to
derive variational algorithms for HMMs, and
Refs.~\cite{beal2003,mackay2003,bishop2006} for more general
discussion of variational inference methods.

\subsubsection{Parameter distributions}
The results of plugging our diffusinve HMM into the parameter update
equation \eqref{seq:qparam} are as follows.  The initial state
probability vector, and each row in the transition matrix (denoted
$A_{j,:}$), are Dirichlet distributed,
\begin{align}
q(\vec{\pi})=&\Dir(\vec{\pi}|\vec{w}^{(\vec{\pi})}),\label{seq:VBpi}\\
w_j^{(\vec{\pi})}=&\tilde w_j^{(\vec{\pi})}+\mean{\delta_{j,s_1}}_{q(s_{1:T})},
\label{seq:VBMpi}\\
%\end{align}\begin{align}
q(A_{i,:})=&\Dir(A_{i,:}|\vec{w}^{(\matris{A})}),\\
w_{ij}^{(\matris{A})}=&\tilde w_{ij}^{(\matris{A})}
+\sum_{t=2}^T\mean{\delta_{i,s_{t-1}}\delta_{j,s_t}}_{q(s_{1:T})}.\label{seq:VBMAj}
\end{align}
Here, variables under tilde's ($\tilde{~}$) are hyperparameters that
parameterize the prior distributions, and can be interpreted as
pseudo-observations. The Dirichlet density function is
\begin{equation}
  \Dir(\vec \pi | \vec u)=\frac{\Gamma(u_0)}{\prod_j\Gamma(u_j)}
  \prod_j \pi_j^{u_j-1},\quad u_j>1,
\end{equation}
where $u_0=\sum_j u_j$ is called the strength, and the density is
non-zero in the region $0\le\pi_j\le1$, $\sum_j\pi_j=1$ . Before
moving on, we quote some useful expectation values for future
reference,
\begin{align}
  \mean{\ln\pi_i}_{q(\vec{\pi})}=&\psi(w_i^{(\vec{\pi})})-\psi(w_0^{(\vec{\pi})}),
  \label{seq:mlogpi}\\
  \mean{\ln A_{ij}}_{q(\matris{A})}=&\psi(w_{ij}^{(\matris{A})})
  -\psi(w_{i0}^{(\matris{A})}),\label{seq:mlogA}
\end{align}
where $\psi(x)$ is the digamma function, and
\begin{align}
  \mean{\pi_i}_{q(\vec{\pi})}=&\frac{w_i^{(\vec{\pi})}}{w_0^{(\vec{\pi})}},\\
  \Var[\pi_i]_{q(\vec{\pi})}=&
  \frac{w_i^{(\vec{\pi})}\big((1-w_i^{(\vec{\pi})}\big)}{
    (w_0^{(\vec{\pi})})^2\big(1+w_0^{(\vec{\pi})}\big)},\\
  \mean{A_{ij}}_{q(\matris{A})}=&\frac{w_{ij}^{(\matris{A})}}{w_{i0}^{(\matris{A})}},\\
  \Var[A_{ij}]_{q(\matris{A})}=&
  \frac{w_{ij}^{(\matris{A})}\big(1-w_{ij}^{(\matris{A})}\big)}{
    (w_{i0}^{(\matris{A})})^2\big(1+w_{i0}^{(\matris{A})}\big)}.
\end{align}
The bead motion parameters have the following variational distributions
\begin{align}\label{seq:KBtrial}
  q(K_j,B_j)=&\frac{B_j^{n_j}}{W_j}e^{-B_j\big(v_j(K_j-\mu_j)^2+c_j\big)},\\
  W_j=&\frac{c^{-(n_j+\frac 12)}\Gamma(n_j+\frac 12)}{\sqrt{v_j/\pi}},
\end{align}
with the range $B_j\ge 0$, $-\infty<K_j<\infty$. Physically, we might
rather expect $0<K_j<1$, but the extended range for $K_j$ simplifies
the calculations a lot. The VBM equations are
\begin{align}
  n_j=& \tilde n_j+M_j,\label{seq:nj}\\
  c_j=& \tilde c_j+C_j+\tilde v_j\tilde\mu_j^2
  -\frac{\big(\tilde v_j\tilde\mu_j+U_j\big)^2}{\tilde v_j+V_j},\label{seq:cj}\\
  v_j=&\tilde v_j+V_j,\label{seq:vj}\\
  \mu_j=&\frac{\tilde v_j\tilde\mu_j+U_j}{\tilde v_j+V_j},\label{seq:muj}\\
\end{align}
with
\begin{align}
  M_j=&\sum_{t=2}^T\mean{\delta_{s_t,j}},\label{seq:Mj}\\
  C_j=&\sum_{t=2}^T\mean{\delta_{s_t,j}}\x_t^2,\label{seq:Cj}\\
  V_j=&\sum_{t=2}^T\mean{\delta_{s_t,j}}\x_{t-1}^2.\label{seq:Vj}\\
  U_j=&\sum_{t=2}^T\mean{\delta_{s_t,j}}\x_t\cdot\x_{t-1},\label{seq:Uj}
\end{align}
Some useful expectation values for future reference are
\begin{align}
  \mean{\ln B_j}_{q(\vec{B},\vec{K})}=&\psi\big(n_j+\frac 12\big)-\ln c_j,
  \label{seq:meanBlogB}\\
  \mean{B_j}_{q(\vec{B},\vec{K})}=&\frac{n_j+\frac 12}{c_j},\\
  \mean{B_jK_j^2}_{q(\vec{B},\vec{K})}=&\frac{1}{2v_j}
  +\mu_j^2\frac{n_j+\frac 12}{c_j},\\
  \mean{B_jK_j}_{q(\vec{B},\vec{K})}=&\mu_j\frac{n_j+\frac 12}{c_j},
  \label{seq:meanbK2BK}\\
  \text{Var}[B_j]_{q(\vec{B},\vec{K})}=&\frac{n_j+\frac 12}{c_j^2},\\
  \mean{K_j}_{q(\vec{B},\vec{K})}=&\mu_j,\\ 
  \text{Var}[K_j]_{q(\vec{B},\vec{K})}=&\frac{c_j}{2v_j(n_j-\frac 12)}.
\end{align}
\subsubsection{Hidden state distribution}
The variational distribution has a simple form,
\begin{equation}\label{seq:qgt}
  \ln q(s_{1:T})=-\ln Z+\sum_{t=1}^T\ln h_{s_t}(t)+\sum_{t=2}^T\ln J_{s_{t-1},s_t},
\end{equation}
i.e., an initial state distribution, a point-wise term that depends on
the initial conditions and the data, and a transition
probability. The point-wise 
\begin{equation}
  \ln q(s_{1:T})=-\ln Z+\sum_{t=1}^T\ln h_{s_t}(t)+\sum_{t=2}^T\ln J_{s_{t-1},s_t},
\end{equation}
i.e., an initial state distribution, point-wise terms that depends on
the initial conditions and the data, and transition terms. The
mathematical form of this expression is the same as encountered in
maximum-likelihood optimization of hidden Markov Models, and hence the
normalization constant and expectation values needed for the parameter
update equations can be computed by the Baum-Welch algorithm
\cite{baum1970}, which resembles the transfer matrix solution for spin
models in statistical physics. 

Similarly, and the most likely sequence of hidden states can be
computed by the Viterbi algorithm \cite{viterbi1967}.

Specifically, the initial term is given by
\begin{equation}
  \ln h_j(1)=
  \mean{p(s_1=j|\vec{\pi})}_{q(\vec{\pi})}
  =\psi(w_{j}^{(\vec{\pi})})-\psi(w_0^{(\vec{\pi})}),
\end{equation}
the point-wise contributions for $t>1$ are
\begin{multline}
  \ln h_j(t)=
  \psi\big(n_j+\frac12\big)-\ln(\pi c_j)-\frac{\x_{t-1}^2}{2v_j}\\
-\frac{n_j+\frac12}{c_j}\Bigg(
\x_{t-1}^2\bigg(\mu_j-\frac{\x_t\cdot\x_{t-1}}{\x_{t-1}^2}\bigg)^2\\
+\x_t^2-\frac{\big(\x_t\cdot\x_{t-1}\big)^2}{\x_{t-1}^2}\Bigg),
\end{multline}
and the transition terms are given by
\begin{equation}
    \ln J_{ji}=\psi\big(w_{j,i}^{(\matris{A})}\big)
    -\psi\big(\sum_{k=1}^Nw_{j,k}^{(\matris{A})}\big).
\end{equation}
\subsection{VBEM iterations and model search}
The iterative optimization of the variational distributions are done
as follows. To start with, an initial guess for the variational
parameter distributions are generated. We then alternate between VBE
step, in which we construct the hidden state distribution and compute
the averages $\mean{\delta_{j,s_t}}_{q(s)}$ and
$\mean{\delta_{j,s_t}\delta_{k,s_{t+1}}}_{q(s)}$ in a Baum-Welch
forward-backward sweep, and a VBM step, in which we use these averages
to update the parameter variational distributions, until the lower
bound converges.

The variational approach has the additional useful tendency to
penalizing overfitting already during the VBEM iterations, by
depopulating superfluous states
\cite{mackay1997,beal2003,persson2013}. We exploit this property by
using a greedy search algorithm to explore the model space. The basic
strategy is to start by fitting a model with many states from random
initial conditions, and then exploring less complex models by
gradually removing the least populated states. This saves computing
time by supplying good initial guesses for the low complexity models
(which therefore converge quickly), and by lowering the number of
independent restarts, since it is easier to construct a good initial
guess for a model with many states.

\subsection{The lower bound}
has an especially simple form just after the VBE step
\cite{mackay1997,beal2003,persson2013}, given by the normalization
constant $\ln Z$ of the variational hidden state distribution, minus
the Kullback-Leibler divergences between the variational and prior
parameter distributions,
\begin{multline}
  F=\ln Z
  -\int d\vec{\pi} q(\vec{\pi}) \ln\frac{q(\vec{\pi})}{p(\vec{\pi})}\\
  -\sum_{j=1}^N\Bigg[
  \int d^NA_{j,:}\;q(A_{j,:})\ln\frac{q(A_{j,:})}{p_0(A_{j,:})}\\
  +\int dB_jdK_j\;q(B_j,K_j)\ln\frac{q(B_j,K_j)}{p_0(B_j,K_j)}
  \Bigg].
\end{multline}
The Kullback-Leibler terms can be expressed in terms of the
expectation values computed above. For the initial state distribution,
we get
\begin{multline}
  \int d\vec{\pi} q(\vec{\pi}) \ln\frac{q(\vec{\pi})}{p_0(\vec{\pi})}
  =\ln\tilde w_0^{(\vec{\pi})}
    -\psi(\tilde w_0^{(\vec{\pi})})-\frac{1}{\tilde w_0^{(\vec{\pi})}}\\
    +\sum_{j=1}^N\left[
      \big(w_j^{(\vec{\pi})}-\tilde w_j^{(\vec{\pi})}\big)\psi(w_j^{(\vec{\pi})})
      -\ln\frac{\Gamma(w_j^{(\vec{\pi})})}{\Gamma(\tilde w_j^{(\vec{\pi})})}
      \right].
\end{multline}
To get this simple form, we used that $w_0^{(\vec{\pi})}=1+\tilde
w_0^{(\vec{\pi})}$ (since $\sum_j\mean{\delta_{j,s_1}}=1$), and the
identities $\Gamma(x+1)=x\Gamma(x)$ and $\psi(x+1)=\psi(x)+\frac 1x$.
Furthermore, each row of the transition probability matrix contributes
\begin{multline}
  \int d^NA_{j,:}\;q(A_{j,:})\ln\frac{q(A_{j,:})}{p_0(A_{j,:})}\\
  =\ln\frac{\Gamma(w_{j0}^{(\matris{A})})}{\Gamma(\tilde w_{j0}^{(\matris{A})})}
    -(w_{j0}^{(\matris{A})}-\tilde w_{j0}^{(\matris{A})})\psi(w_{j0}^{(\matris{A})})\\
    -\sum_{k=1}^N\bigg[
      \ln\frac{\Gamma(w_{jk}^{(\matris{A})})}{\Gamma(\tilde w_{jk}^{(\matris{A})})}
      -\big(w_{jk}^{(\matris{A})}-\tilde w_{jk}^{(\matris{A})}\big)
      \psi(w_{jk}^{(\matris{A})})\bigg].
\end{multline}
Finally, the emission parameter of each state contributes
\begin{multline}\label{seq:KBKL}
  \int dB_j\int d^NK_j\;q(B_j,K_j)
  \ln\frac{q(B_j,K_j)}{p(B_j,K_j)}=\ldots\\
  =-\frac{n_j+\frac 12}{c_j}
    \Big(c_j-\tilde c_j-\tilde v_j(\mu_j-\tilde \mu_j)^2\Big)\\
    +\frac 12\ln\frac{v_j}{\tilde v_j}
    +(\tilde n_j+\frac 12)\ln\frac{c_j}{\tilde c_j}
    -\ln\frac{\Gamma\big(n_j+\frac 12\big)}{\Gamma\big(\tilde n_j+\frac 12\big)}\\
    +(n_j-\tilde n_j)\psi\big(n_j+\frac 12\big)
    +\frac{\tilde v_j}{2v_j}
    -\frac 12.
\end{multline}

\subsection{Two types of states}
The above algorithm is readily extended to treat the model where
genuine and spurious states are separated into two different hidden
processes. We implemented a brute force approach to this problem,
where we define new composite hidden states $\hat s_t=(s_t,c_t)$ and
run the above algorithm on this composite model. This has a
significant computational cost, since a simple model with $N_{gen.}$
genuine states and $N_{sp.}$ spurious ones gets
$N_{gen.}\times(1+N_{sp.})$ states after conversion. However, since we
do not perform exhaustive model search in this representation and can
utilize the simpler model to make good initial guesses, this is not a
significant problem.

\subsection{Choice of prior distributions}
We would like to choose uninformative prior distributions in order to
let the data speak for itself as much as possible. This is
unproblematic for the emission parameters $K,B$, since the amount of
data in all states is large enough to overwhelm any prior
influence. We use
\begin{align}
  \tilde\mu_j=&0.6, &
  \tilde n_j=&1,\\
  \tilde v_j =&5.56\text{ nm$^2$},&
  \tilde c_j=&30000\text{ nm$^2$},
\end{align}
which corresponds to 
\begin{align}
  \mean{K_j}=&0.6,&
  \mean{B_j}=&5\times 10^{-5}\text{ nm$^{-2}$},\\
  \mathrm{std}(K_j)=&0.3,&
  \mathrm{std}(B_j)=&141.4\times 10^{-5}\text{ nm$^{-2}$}.
\end{align}
The initial state prior is unproblematic for the opposite reason: the
long length of the trajectories makes the initial state relatively
unimportant to describe the data. We use a constant prior strength of
5,
\begin{equation}
  \tilde w_j^{(\vec\pi)}=5/N,
\end{equation}
where $N$ is the number of hidden states.

The transition probabilities needs more care, because the potentially
low number of transitions per trajectory makes the prior relatively
more influential.  Following \citet{persson2013}, we parameterize this
prior in terms of an expected mean dwell time and an overall number of
pseudocounts (prior strength) for each hidden state.  In particular,
we define a transition \textit{rate} matrix Q with mean dwell time
$t_D$,
\begin{equation}
  Q_{ij}=\frac{1}{t_D}\left(-\delta_{ij}
  +\frac{1-\delta_{ij}}{N-1}\right),
\end{equation}
and then construct the prior based on the transition probability
propagator per unit time step,
\begin{equation}
  \tilde w_{ij}^{(\matris{A})}=\frac{t_Af_{sample}}{n_{downsample}}
  e^{\Delta tQ}.
\end{equation}
Here, $t_A$ is the prior strength; both $t_A$ and $t_D$ is specified
in time units to be invariant under a change of sampling
frequency. Further, the timestep is given by $\Delta
t=n_{downsample}/f_{sample}$, where $f_{sample}$ is the sampling
frequency (30 Hz in our case), and $n_{downsample}$ is the
downsampling factor (we use 3).

Numerical experiments by \citet{persson2013} show that choosing the
strength too low compared to the mean dwell time produces a bias
towards sparse transition matrices. This is not desireable in our
case, and we therefore use $t_D=1$ s, and $t_A=5$ s throughout this
work.


{\bf Prior for factorial model: TBA}.\cite{factorialmodelprior}

\subsection{Empirical Bayes update equations}
The empirical Bayes update equations optimizes the lower bound with
respect to the hyperparameters in the prior distribution. This means
optimizing sums of Kullback-Leibler divergence terms. 

The initial state probability, and the rows of the transition
probability matrix, are both Dirichlet distributed. Thus, for $M$
trajectories with Dirichlet parameters $u_j^{(i)}$, $i=1,2,\ldots,M$,
and hyperparameters $\tilde{u}_j$ ($u = w^{(\vec\pi)},u^{(\matris{A})}$),
we need to solve
\begin{multline}
\frac{d}{d \tilde u_j} \sum_i\Bigg( 
 \ln\frac{\Gamma(u_0^{(i)})}{\Gamma(\tilde u_0)}
    -(u_0^{(i)}-\tilde u_0)\psi(u_0^{(i)})\\
    -\sum_{k=1}^N\bigg[
      \ln\frac{\Gamma(u_k^{(i)})}{\Gamma(\tilde u_k^{(i)})}
      -\big(u_k^{(i)}-\tilde u_k\big)
      \psi(u_k^{(i)})\bigg]\Bigg)=0,
\end{multline}
where $u_0^{(i)}=\sum_ku_k^{(i)}$ and similar for $\tilde u_0$. This
leads to the update equations
\begin{equation}
  \psi(\tilde u_0)-\psi(\tilde u_j)
=\frac{1}{M}\sum_i\bigg(
  \psi(u_0^{(i)})-\psi(u_j^{(i)})\bigg).
\end{equation}
A numerical solution turned out to be easier using the variables
$\tilde U_j=\ln \tilde u_j$ (to numerically enforce $\tilde u_j>0$).

For the emission parameters, the update equations are instead derived
from minimizing \Eq{seq:KBKL} summed over $M$ trajectories,
\begin{multline}
  f_{KB}=\sum_i\Bigg(
  -\frac{n^{(i)}+\frac 12}{c^{(i)}} \Big(c^{(i)}-\tilde c-\tilde v(\mu^{(i)}-\tilde
  \mu)^2\Big)\\ +\frac 12\ln\frac{v^{(i)}}{\tilde v} +(\tilde
  n+\frac 12)\ln\frac{c^{(i)}}{\tilde c}
  -\ln\frac{\Gamma\big(n^{(i)}+\frac 12\big)}{\Gamma\big(\tilde n+\frac
    12\big)}\\ +(n^{(i)}-\tilde n)\psi\big(n^{(i)}+\frac 12\big)
  +\frac{1}{2}\Big(\frac{\tilde v}{v^{(i)}} -1\Big)\Bigg).
\end{multline}
Minimizing with respect to $\tilde\mu$ and $\tilde v$ leads to
\begin{align}
   \tilde\mu=&\frac{1}{M}\sum_i\mu^{(i)},\\
   \frac{1}{\tilde v}=&
   \frac{1}{M}\sum_i\Big(\frac{1}{v^{(i)}}+2(\tilde\mu-\mu^{(i)})^2\Big).
\end{align}
The remaining $\tilde c$ and $\tilde n$ lead to
\begin{align}
  \frac{\tilde n+\frac 12}{\tilde c}=&
  \frac{1}{M}\sum_i\frac{n^{(i)}+\frac 12}{c^{(i)}},\\
  \ln\tilde c-\psi\big(\tilde n+\frac 12\big)=&
  \frac{1}{M}\sum_i\bigg(
\ln c^{(i)}-\psi\big(n^{(i)}+\frac 12\big)\bigg),
\end{align}
which we solve numerically. This gets easier by defining
$\alpha=\frac{\tilde n+\frac 12}{\tilde c}$, then solve the second
equation for $\tilde c$ numerically, and finally compute $\tilde
n=\alpha\tilde c-\frac 12$.

\subsection{Factorial model}
\subsection{Notation and symbols}\label{sec:notation}
vbSPT stores mathematical objects in matlab structures that contain
parameters for the variational and prior distributions, and various
other things. In this section we list some of them.

First, \verb+VB7_batch_manage+ with the collect option returns a
filename, and cell vectors of structs that contain results of the
model search for each trajectory, indexed as the filenames in the
runinput file, e.g. \verb+trj{k}{j}+ contains the analysis result for
\verb+looping_filename{k}{j}+ etc. 

Most importantly, teh Wtrj and Wcal fields are the converged model
structs, whose content are detailed below (using W as the generic
model name). In addition the columns of the arrays
\verb+NFtrj+/\verb+NFcal+ contain $N,\hat N, F, iter$ for each model
that was converged during the nodel search, and $iter$ is the restart
number that produced it. In \verb+NFitrj/NFical+, the best model for
each size is listed in the same way, with the last column indicating
the rows in \verb+NFtrj,NFcal+ where these optimal models can be
found. 

Further details about the model structs are given in tables
\ref{tab:Wfields}-\ref{tab:Westimates}.

\begin{table}
\caption{Fields in a model object $W$.}\label{tab:Wfields}
\begin{center}\begin{tabular}{|l|l|}
\hline
field & \\
\hline
\hline
    \verb+W.N+& \parboxc{$N$, number of (genuine) states.} \\
    \hline
    \verb+W.Nc+& \parboxc{       
      Number of indicator states $\hat N$. $\hat N=1$ means no spurious
      states, i.e., the simple HMM.}\\ 
    \hline
    \verb+W.F+& \parboxc{Lower bound $F$.} \\
    \hline
\end{tabular}\end{center}
\end{table}

\begin{table}
\caption{Representration of variational distributions for genuine
  states in a model object named $W$.}\label{tab:Qparameters}
\begin{center}\begin{tabular}{|l|c|c|}
    \hline
    field & symbol & Eq.\\
    \hline\hline
    \verb+W.M.wPi(j)+& $ w_j^{(\vec{\pi})}$    & \\ 
    \verb+W.PM.wPi(j)+& $\tilde w_j^{(\vec{\pi})}$    & \eqref{seq:VBMpi}\\ 
    \verb+W.E.ds_1(j)+& $\mean{\delta_{j,s_1}}$ & \\ 
    \hline
    
    \verb+W.PM.wA(i,j)+& $\tilde w_{ij}^{(\matris{A})}$& \\ 
    \verb+W.M.wA(i,j)+& $w_{ij}^{(\matris{A})}$& \eqref{seq:VBMAj}\\ 
    \verb+W.E.wA(i,j)+& $\sum_{t=2}^T\mean{\delta_{i,s_{t-1}}\delta_{j,s_t}}$
    & \\ 
    \hline
    \verb+W.PM.n(j)+& $\tilde n_j$& \eqref{seq:nj}\\ 
    \verb+W.M.n(j)+& $n_j$& \eqref{seq:nj}\\ 
    \verb+W.E.M(j)+& $M_j$& \eqref{seq:Mj}\\ 
    \hline
    \verb+W.PM.c(j)+& $\tilde c_j$& \eqref{seq:cj}\\ 
    \verb+W.M.c(j)+& $c_j$& \eqref{seq:cj}\\ 
    \verb+W.E.C(j)+& $C_j$&\eqref{seq:Cj}\\
    \hline
    \verb+W.PM.v(j)+& $\tilde v_j$& \eqref{seq:vj}\\ 
    \verb+W.M.v(j)+& $v_j$& \eqref{seq:vj}\\ 
    \verb+W.E.V(j)+& $V_j$& \eqref{seq:Vj}\\ 
    \hline
    \verb+W.PM.mu(j)+& $\tilde \mu_j$& \eqref{seq:muj}\\
    \verb+W.M.mu(j)+& $\mu_j$& \eqref{seq:muj}\\  
    \verb+W.E.U(j)+& $U_j$&\eqref{seq:Uj}\\
    \hline
  \end{tabular}\end{center}\end{table}
\begin{table}
\caption{Representration of variational distributions for indicator
  states $c_t$ in a model object named $W$. Fields relating to the
  emission model (.n, .c, .v, .mu, etc.) have the same meaning as for
  the genuine states $s_t$, except that their first element is not
  used, since $c_t=1$ indicate a genuine state.}\label{tab:QSparameters}
\begin{center}  
  \begin{tabular}{|l|c|c|}
    \hline
    field & symbol & Eq.\\
    \hline\hline
%
    \hline
    \verb+W.Mc.wPi(j)+& $w_j^{(\vec{\pi})}$    & \\ 
    \verb+W.PMc.wPi(j)+& $\tilde w_j^{(\vec{\pi})}$    & \eqref{seq:VBMcpi}\\ 
    \verb+W.Ec.ds_1(j)+& $\mean{\delta_{j,c_1}}$ & \\ 
    \hline
     \verb+W.PMc.wA(j)+& & \\ 
    \verb+W.Mc.wA(j)+& & \eqref{seq:VBMcAj}\\ 
    \verb+W.Ec.wA(j)+& &\\
    \hline
     \verb+W.PMc.wR(i,j)+& & \\ 
    \verb+W.Mc.wR(i,j)+& & \eqref{seq:VBMcRj}\\ 
    \verb+W.Ec.wR(i,j)+&& \\
    \hline
  \end{tabular}
\end{center}
\end{table}
   
 \begin{table*}
\caption{Selected fields that characterize converged models (named
  W). The fields W.est and W.est2 are constructed by VB7\_VBEMiter.m
  (although W.est2 must be specifically requested), and fields not
  mentioned here can be looked up there. Averages are
  w.r.t. variational parameter distributions unless stated otherwise.}\label{tab:Westimates}
\begin{center}  
  \begin{tabular}{|l|l|}
    \hline   field & comment \\  \hline  
\hline
\verb+W.est.sAverage+&
\parboxcc{Occupation probability of genuine states $s_t$, computed by
  classification, i.e., sAverage(j) proportional to
  $\sum_t\mean{\delta_{j,s_t}}$.}\\
\hline
\verb+W.est.cAverage+&
\parboxcc{Occupation probability of indicator states
  $c_t$, by classification.}\\
\hline
\verb+W.est.sVisible+&
\parboxcc{Occupation probability of genuine states $s_t$, by classification that excludes spurious states, i.e., sAverage(j) proportional to $\sum_t\mean{\delta_{j,s_t}\delta_{1,c_t}}$.}\\
\hline
\hline
\verb+W.est.A+&
\parboxcc{Mean transition probabilities for genuine states,
  $\mean{\matris{A}}_{q(\matris{A})}$.}\\
\hline
\verb+W.est.dA+&
\parboxcc{Standard devition of stransition probability matrix
  $\matris{A}$.}\\
\hline
\verb+W.est.tD+&
\parboxcc{Mean dwell times of genuine states in units of time,
  computed from the elements of $\mean{\matris{A}}$.}\\
\hline
\verb+W.est.lnQss+&\parboxcc{Log average transition probabilities, goes into
  $q(s_{1:T})$.}\\
\hline
&\parboxcc{Corresponding averages for spurious state distributions are
  also computed, Ac, dAc, Rc, dRc, tStick, lnQcc, lnQsc, tStick, tUnstick.}\\
\hline\hline
\verb+W.est.sKaverage(j)+&$\mean{K_j}=\mu_j$.\\\hline
\verb+W.est.sBaverage(j)+&$\mean{B_j}$.\\\hline
\verb+W.est.sRMS(j)+&
$RMS_j=\sqrt{\mean{\x_t^2|s_t=j}}
\approx (\mean{B_j}(1-\mean{K_j}^2))^{-\frac 12}$.\\\hline
\verb+W.est.sTC(j)+&Approx. correlation time 
  $\tau_j\approx -\Delta t/\log\mean{K_j}$, units of time.\\\hline
\verb+W.est.sKstd(j)+&Standard deviation of $K_j$.\\\hline
\verb+W.est.sBstd(j)+&Standard deviation of $B_j$.\\\hline
\verb+W.est.cXXX+&Corresponding properties of spurious
  states are named with $s\to c$.\\ \hline\hline
\verb+W.est2.qt+& 
\parboxcc{State occupancy probability for combined states
  $(s_t,c_t)$. Use sMap and cMap to extract genuine/spurious
  occupancies, e.g.,\\ $p(s_t=j)=$sum(W.est2.qt(t,:).*(W.est.sMap==j)).}\\\hline
\verb+W.est2.sMaxP(t)+& Most likely genuine state at time $t$.\\
\verb+W.est2.cMaxP(t)+& Most likely indicator state at time $t$.\\
     % Viterbi paths. 
     % W.est2.zViterbi=uint8(VBlogViterbi(lnQ,lnqst,Q,qst));
%     \verb+W.est2.zViterbi=uint8(VBviterbi_log(lnQ,lnqst)); % faster!     
\verb+W.est2.sViterbi+& Viterbi path (most likely sequence of states) for $s_t$.\\
\verb+W.est2.cViterbi+& Viterbi path (most likely sequence of states) for $c_t$.\\\hline

&\parboxcc{W.est2 also contains a few other intermediate fields from
  the VBEM iteration that are mainly good for debugging. This
  substructure is thus very bulky and somewhat expensive to compute,
  which is the reason computing it is optional. }\\\hline
\end{tabular}\end{center}
\end{table*}  
\clearpage
\subsection{Doing empirical Bayes}
Our empirical Bayes analysis of multiple models was not implemented as
part of our analysis pipeline, but instead ran using tailored
scripts. These are not included with vbTPM, but the optimization
procedures for the individual prior distributions are included, in
\verb+VB7_EBupdate_dirichlet+, \verb+VB7_EBupdate_KB+, and
\verb+VB7_EBupdate_KB2+.

\bibliographystyle{unsrtnat}
\bibliography{references}%,jwm_single_molecule,jwm_machine_learning}

\end{document}
